\documentclass[11pt]{article}
\usepackage[margin = 2.5cm, headheight = 36pt]{geometry}
\usepackage{fancyhdr}

\usepackage{amssymb}
\usepackage{mathtools}

% \usepackage{showlabels}

\pagestyle{fancy}
\fancyhead{}
\fancyhead[L]{Day 07}
\fancyhead[R]{AoC 2021}
\fancyfoot{}
\fancyfoot[R]{\thepage}

\renewcommand{\baselinestretch}{1.15}
\setlength{\parindent}{0cm}

\begin{document}

\section*{Set-up}

Let \(X\) be a sorted list of positive integers with length \(L\), then
\begin{equation*}
    X = {
    [x_0, x_1, \ldots, x_{L - 2}, x_{L - 1}]:
    \{x_i, i \in \mathbb{Z}: x_i, i \geqslant 0\}
    }
\end{equation*}

For a list \(T\), let \(s(T)\) be the sum of all the elements in the list, then
\begin{equation*}
    s(T) = \sum_{i = 0}^{L_T - 1} t_i: \{i \in \mathbb{Z}: i \geqslant 0\} {,}
\end{equation*}
where \(L_T\) is the length of \(T\) and \(t_i\) is the element at index \(i\)
in \(T\).

\section*{Part 1}
The goal is to derive a function \(p(k)\) such that:
\begin{equation}
    p(k) = \sum_{i = 0}^{L - 1} |x_i - k| \label{eq:1}
\end{equation}

Let \(f(i, k) = |x_i - k|\), then
\begin{equation*}
    f(i, k) =
    \begin{dcases}
        x_i - k: & x_i - k \geqslant 0 \\
        k - x_i: & x_i -k < 0
    \end{dcases}
\end{equation*}

\vspace{0.5\baselineskip}

For each value of \(k\), split \(X\) into \(X_l\) and \(X_h\) such that \(X_l\)
contains all \(x_i: x_i < k\) and \(X_h\) contains all \(x_i: x_i \geqslant
k\). Let the lengths of \(X_l\) and \(X_h\) be \(L_l\) and \(L_h\),
respectively, then
\begin{align}
    L_l + L_h       & = L \iff L_h = L - L_l \label{eq:2}             \\
    s(X_l) + s(X_h) & = s(X) \iff s(X_h) = s(X) - s(X_l) \label{eq:3}
\end{align}

Thus, \eqref{eq:1} can be rewritten in terms of \(X_l\) and \(X_h\) as follows:
\begin{equation*}
    \setlength{\jot}{0.5\baselineskip}
    \begin{aligned}
        p(k) & = {
        \sum_{i = 0}^{L_l - 1} (k - {x_{l}}_{i}) +
        \sum_{i = 0}^{L_h - 1} ({x_{h}}_{i} - k):
        \{{x_{l}}_{i} \in X_l, {x_{h}}_{i} \in X_h\}
        }                                        \\
        %
        p(k) & = L_l k - s(X_l) + s(X_h) - L_h k
    \end{aligned}
\end{equation*}

From \eqref{eq:2} and \eqref{eq:3},
\begin{align}
    p(k) & = {
    L_l k - s(X_l) +
    \underbrace{s(X) - s(X_l)}_{s(X_h)} -
    \underbrace{(L - L_l)}_{L_h} k
    } \nonumber \\
    p(k) & =  {
    s(X) + 2 L_l k - (2 s(X_l) + L k)
    \tag{P} \label{eq:p}
    }
\end{align}

\newpage

\section*{Part 2}

Let \(S_N(n)\) be the arithmetic series with \(a = 1\) and \(d = 1\), then
\begin{equation*}
    S_N(n) = \frac{1}{2} \left({n^2 + n}\right)
\end{equation*}


The goal is to derive a function \(q(k)\) such that:
\begin{equation*}
    q(k) = \sum_{i = 0}^{L - 1} S_n(|x_i - k|)
\end{equation*}

Using the definition of \(S_N\),
\begin{equation*}
    \setlength{\jot}{0.5\baselineskip}
    \begin{aligned}
        q(k) & = \frac{1}{2} \sum_{i = 0}^{L - 1}
        \left({(x_i - k)^2 + |x_i - k|}\right)    \\
        %
        q(k) & = \frac{1}{2} \left({
            \sum_{i = 0}^{L - 1} (x_i - k)^2 +
            \underbrace{
                \sum_{i = 0}^{L - 1} |x_i - k|
            }_{p(k)}
        }\right)                                  \\
        %
        q(k) & = \frac{1}{2} \left({
                    \sum_{i = 0}^{L - 1} \left({x_i^2 + k^2 - 2 x_i k}\right) +
                    p(k)
        }\right)                                  \\
        %
        q(k) & = \frac{1}{2} \left({
                    \sum_{i = 0}^{L - 1} \left({x_i^2}\right) +
                    p(k) + L k^2
                }\right) - k s(X)
    \end{aligned}
\end{equation*}

For a list \(T\), let \(ss(T)\) be the sum of the squares of all the elements
in the list, then
\begin{equation*}
    ss(T) = {
    \sum_{i = 0}^{L_T - 1} {t_i}^2: \{i \in \mathbb{Z}:
    i \geqslant 0\} {,}
    }
\end{equation*}
where \(L_T\) is the length of \(T\) and \(t_i\) is the element at index \(i\)
in \(T\); then,

\begin{equation*}
    q(k) = \frac{p(k) + ss(X) + L k^2}{2} - k s(X)
    \tag{Q} \label{eq:q}
\end{equation*}

\end{document}
